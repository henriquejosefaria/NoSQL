\section{Introdução}

A migração entre bases de dados é uma tarefa relativamente típica quando se pretende obter maior eficiência em certas consultas ou melhorar o desempenho de forma geral. As diferentes base de dados sejam estas SQL ou NoSQL têm o seu propósito. Normalmente os serviços maiores, tem uma base de dados relacional para "suportar" todo o serviço e outras NoSQL para "suportar" consultas mais especificas com maior rapidez.

O objetivo do trabalho passou por realizar então uma migração de uma base de dados MySQL para OracleDB, uma parte desta para MongoDB e para Neo4j.

As seções sobre cada base de dados seguem o mesmo princípio, isto é, a motivação. Note-se que algumas modificações feitas(Oracle - Contém uma seção sobre as ficuldades com a migração).

Primeiramente, discute-se sobre a base de dados relacional, seguindo com a documental e, finalmente, a base de dados gráfica. 

Por fim terminamos o relatório com algumas conclusões sobre as base de dados desenvolvidas.


\section{OracleDB}

\subsection{Criação da base de dados}

\subsubsection{Motivação}\hfill
\newline
\par A criação da base de dados Oracle pretende responder á necessidade de escalar horizontalmente a nossa base de dados bem com garantir maior flexibilidade e custos reduzidos na manutenção.

\subsubsection{Criação - modificações}\hfill
\newline

\par A criação destas bases de dados passa por um script muito identico ao do MySQL.\newline
\par Ao criar esta base de dados os tipos TINYINT e SMALLINT foram convertidos para NUMBER(10) visto que estes tipos não têm representação nos DataTypes Oracle. Para além disso o tipo BOOLEAN foi modificado visto que as bases de dados Oracle também não suportam este tipo de dados sugerindo optar por um VARCHAR2(Y,N) ou, como foi impementado, NUMBER(1,0).
\par O tipo DATETIME do MySQL foi transformado no tipo correspondente na base de dados Oracle DATE e o tipo VARCHAR foi convertido em VARCHAR2. Por fim o tipo DECIMAL(5,2) presente na base de dados foi convertido para o seu correspondente Oracle FLOAT(2).

\subsection{Criação - Dificuldades na implementação}

\par Ao implementar a mudança da base de dados em python surgiram alguns imprevistos.
\par Em primeiro lugar a ligação á base de dados foi difícil de conseguir visto que após instalada a base de dados tivemos de proceder á instalação do instant-client, isto suscitou problemas pois este está na versão 19.3.0.0.0 e os torturiais para a instalação deste estão desatualizados estando atualmente na versão 12. Isto levou-nos a procurar soluções em foruns, no git e no StackOverflow até que finalmente as variáveis com o caminho foram criadas e os ficheiros do Instant Client foram colocados na diretoria correta.\newline
\par Posteriormente surgiram problemas ao correr os scripts para a criação das tabelas pois estes continham pontuação como ';' no fim dos comandos coisa que não pode ser usada nestes scripts no python, removendo os mesmos foi o suficiente para os scripts de criação de tabelas começarem a funcionar.\newline
\par Outro problema que surgiu estava ligado á criação de duas tabelas que continham chaves estrangeiras uma da outra. Por estarmos a executar o script no python como uma das tabelas não tinha sido criada ao declarar uma chave estrangeira dessa tabela o comando não funcionava. Para resolver este problema, foram criadas primeiro as duas tabelas e a segunda contendo a chave estrangeira referenciando a primeira. Posteriormente foram ambas preenchidas, primeiro a tabela sem chaves estrangeiras seguindo-se a restante. Após serem preenchidas procedeu-se á alteração da primeira tabela criando a restrição de chave estrangeira correspondente.\newline
\par Por fim surgiu outro problema ao guardar os dados. Na tabela Film os dados referentes a special\_features são devolvidos como tuplos em vez de uma string. Para isso foi necessário executar um ciclo for sobre os dados retirados da tabela MySQL e transformar esse tuplos em strings para poderem ser guardados na nossa base de dados.\newline  


\subsection{Preenchimento da Base de Dados}

\par Para preencher a base de dados foram usados os seguintes comandos (para exemplificar vão ser listados os comandos usados na tabela language):\newline
\begin{itemize}
\item \textbf{language\_sql = \textit{"SELECT * FROM language"}}


\par Este primeiro comando serve para criar a query que servirá para ler a tabela language. \newline

\item \textbf{mycursor.execute(language\_sql)}


\par Este comando executa no \textit{MySQL} a leitura da tabela language.\newline

\item \textbf{languageRows = mycursor.fetchall()}


\par Este comando vai buscar todos os registos lidos previamente com a query \textit{language\_sql}.\newline

\item \textbf{oracleCursor.bindarraysize = len(languageRows)}


\par Este comando serve apenas para que o python saiba quantas vezes terá de executar a query que se segue.\newline



\item \textbf{oracleCursor.executemany('insert into LANGUAGE(LANGUAGE\_ID, NAME,LAST\_UPDATE) values(:1,:2,:3)', languageRows)}


\par Esta query será executada sobre cada linha da talela languageRows.\newline

\item \textbf{con.commit()}


\par Este comando permite finalizar os processos realizados adicionando as modificações realizadas por estes permanentemente á nossa base de dados Oracle.


\end{itemize}

\textit{Nota: Estes comandos foram usados em cada tabela copiada de MySQL para OracleDB.}

\subsection{Querys á Base de Dados}

Em seguida listam-se três das querys feitas á base de dados Oracle e as respetivas respostas.


\begin{lstlisting}[caption=Query á base de dados Oracle para devolver todos os Filmes com id inferior ou igual a 3]
oracleCursor.execute("SELECT * FROM FILM WHERE FILM_ID <= 3").fetchall()
\end{lstlisting}


\begin{lstlisting}[=caption=Query á base de dados Oracle para devolver todos os Filmes com o nome das respetivas linguagens]
oracleCursor.execute("SELECT FILM.TITLE, LANGUAGE.NAME FROM FILM,LANGUAGE
 WHERE FILM.LANGUAGE\_ID = LANGUAGE.LANGUAGE\_ID").fetchall()
\end{lstlisting}


\begin{lstlisting}[caption=Query á base de dados Oracle para devolver o total faturado por utilizador]
oracleCursor.execute("SELECT CUSTOMER.FIRST_NAME,CUSTOMER.LAST_NAME,
SUM( AMOUNT ) FROM PAYMENT,CUSTOMER 
WHERE PAYMENT.CUSTOMER_ID = CUSTOMER.CUSTOMER_ID 
GROUP BY CUSTOMER.FIRST_NAME,CUSTOMER.LAST_NAME ").fetchall()
\end{lstlisting}












